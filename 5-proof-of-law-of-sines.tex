\question
Theorem~\ref{thm:law-of-sines} will be proved using Figure~\ref{fig:law-of-sines} as a starting point. (\emph{Note}: we'll use side $a$ and angle $\alpha$ but the argument will apply equally well to the other side and angle pairs.)

\begin{figure}[h]
    \centering
    \begin{tikzpicture}
    \tkzDefPoints{0/0/A, 5/2/B, 1/5/C}
    \tkzDefCircle[circum](A,B,C)
    \tkzGetPoint{O} \tkzGetLength{rO}
    \tkzDefPointOnCircle[angle=-45,center=O,radius=\rO pt]
    \tkzGetPoint{P}
    \tkzDefMidPoint(B,C) \tkzGetPoint{M}
    \tkzDrawPolygon[semithick](A,B,C)
    \tkzDrawCircle[semithick](O,A)
    \ifprintanswers
    \tkzDrawSegment[dim={$a$,24pt,above=6pt}](C,B)
    \tkzDrawSegments[semithick,color=brown](O,B O,C)
    \tkzDrawSegment[semithick,color=blue](O,M)
    \tkzDrawPoints(M)
    \tkzMarkSegments[mark=|,color=brown](O,B O,C)
    \tkzMarkSegments[pos=.75,mark=|,color=brown](O,P)
    \tkzMarkSegments[mark=||,color=blue](B,M C,M)
    \tkzMarkRightAngle[color=gray](B,M,O)
    \tkzMarkAngles[size=20pt,mark=none,color=gray](B,A,C B,O,M)
    \tkzLabelAngle[pos=.5](B,O,M){$\alpha$}
    \tkzLabelPoint[above right](M){$M$}
    \else
    \fi
    \tkzDrawSegment[dim={$a$,24pt,above=6pt}](C,B)
    \tkzDrawSegments[semithick,color=brown](O,P)
    \tkzDrawPoints(A,B,C,O,P)
    \tkzLabelPoint[left](A){$A$}
    \tkzLabelPoint[right](B){$B$}
    \tkzLabelPoint[above](C){$C$}
    \tkzLabelPoint[left](O){$O$}
    \tkzLabelAngle[pos=.5](B,A,C){$\alpha$}
    % \tkzLabelSegment[left](A,C){$b$}
    % \tkzLabelSegment[below](A,B){$c$}
    \tkzLabelSegment[pos=.75,color=brown](O,P){$R$}
    \end{tikzpicture}
    \caption{The circumcircle of triangle $ABC$ passes through its three vertices.}
    \label{fig:law-of-sines}
\end{figure}

\begin{parts}

\part[1]
Form triangle $BOC$ by drawing line segments $BO$ and $CO$ on Figure~\ref{fig:law-of-sines}.

\part[1]
Find the measure of $\angle BOC$.

\begin{EnvFullwidth}
\begin{solutionorgrid}[.75in]
By the angle at the centre theorem, $\angle BOC = 2\alpha$.
\end{solutionorgrid}
\end{EnvFullwidth}

\part[2]
Let $M$ be the midpoint of line segment $BC$. Explain why $\angle OMB = \ang{90}$.

\begin{EnvFullwidth}
\begin{solutionorgrid}[1in]
Since $ABC$ is an isosceles triangle, the altitude has the midpoint as its foot.
\end{solutionorgrid}
\end{EnvFullwidth}

\part[2]
Explain why $\angle MOB = \alpha$.

\begin{EnvFullwidth}
\begin{solutionorgrid}[1.5in]
The altitude of an isosceles triangle bisects the angle at the vertex. Thus,
\[
    \angle MOB = \frac{1}{2} \angle BOC = \alpha.
\]
\end{solutionorgrid}
\end{EnvFullwidth}

\part[2]
Using these results, prove Theorem~\ref{thm:law-of-sines}.

\begin{EnvFullwidth}
\begin{solutionorgrid}[2in]
\begin{proof}
Using the earlier results, in triangle $MOB$ we have
\begin{align*}
    \sin(\alpha) &= \frac{a/2}{R} \\
    R\sin(\alpha) &= \frac{a}{2} \\
    \frac{a}{\sin(\alpha)} &= 2R.
\end{align*}
\end{proof}
\end{solutionorgrid}
\end{EnvFullwidth}

\end{parts}
